%%%%%%%%%%%%%%%%%%%%%%%%%%%%%%%%%%%%%%%%%
% Journal Article
% LaTeX Template
% Version 1.4 (15/5/16)
%
% This template has been downloaded from:
% http://www.LaTeXTemplates.com
%
% Original author:
% Frits Wenneker (http://www.howtotex.com) with extensive modifications by
% Vel (vel@LaTeXTemplates.com)
%
% License:
% CC BY-NC-SA 3.0 (http://creativecommons.org/licenses/by-nc-sa/3.0/)
%
%%%%%%%%%%%%%%%%%%%%%%%%%%%%%%%%%%%%%%%%%

%----------------------------------------------------------------------------------------
%	PACKAGES AND OTHER DOCUMENT CONFIGURATIONS
%----------------------------------------------------------------------------------------

\documentclass[twoside,twocolumn]{article}

\usepackage{blindtext} % Package to generate dummy text throughout this template

\usepackage[sc]{mathpazo} % Use the Palatino font
\usepackage[utf8]{inputenc} % Use the UTF-8 encoding
\usepackage[T1]{fontenc} % Use 8-bit encoding that has 256 glyphs
\linespread{1.05} % Line spacing - Palatino needs more space between lines
\usepackage{microtype} % Slightly tweak font spacing for aesthetics

\usepackage[english]{babel} % Language hyphenation and typographical rules

\usepackage[hmarginratio=1:1,top=32mm,columnsep=20pt]{geometry} % Document margins
\usepackage[hang, small,labelfont=bf,up,textfont=it,up]{caption} % Custom captions under/above floats in tables or figures
\usepackage{booktabs} % Horizontal rules in tables

\usepackage{lettrine} % The lettrine is the first enlarged letter at the beginning of the text

\usepackage{enumitem} % Customized lists
\setlist[itemize]{noitemsep} % Make itemize lists more compact

\usepackage{abstract} % Allows abstract customization
\renewcommand{\abstractnamefont}{\normalfont\bfseries} % Set the "Abstract" text to bold
\renewcommand{\abstracttextfont}{\normalfont\small\itshape} % Set the abstract itself to small italic text

\usepackage{titlesec} % Allows customization of titles
\renewcommand\thesection{\Roman{section}} % Roman numerals for the sections
\renewcommand\thesubsection{\roman{subsection}} % roman numerals for subsections
\titleformat{\section}[block]{\large\scshape\centering}{\thesection.}{1em}{} % Change the look of the section titles
\titleformat{\subsection}[block]{\large}{\thesubsection.}{1em}{} % Change the look of the section titles

\usepackage{fancyhdr} % Headers and footers
\pagestyle{fancy} % All pages have headers and footers
\fancyhead{} % Blank out the default header
\fancyfoot{} % Blank out the default footer
\fancyhead[C]{Morphée Light Paper $\bullet$ \today} % Custom header text
\fancyfoot[RO,LE]{\thepage} % Custom footer text

\usepackage{titling} % Customizing the title section

\usepackage{hyperref} % For hyperlinks in the PDF

\usepackage{flushend} % Balances the size of the columns on the last page

%----------------------------------------------------------------------------------------
%	TITLE SECTION
%----------------------------------------------------------------------------------------

\setlength{\droptitle}{-4\baselineskip} % Move the title up

\pretitle{\begin{center}\huge\bfseries} % Article title formatting
\posttitle{\end{center}} % Article title closing formatting
\title{Morphée: a coordinated and automated brokerage layer on top of Filecoin} % Article title
\author{%
%\textsc{John Smith}\thanks{A thank you or further information} \\[1ex] % Your name
\textsc{}
\normalsize Polyphene \\ % Your institution
\normalsize \href{mailto:contact@polyphene.io}{contact@polyphene.io} % Your email address
%\and % Uncomment if 2 authors are required, duplicate these 4 lines if more
%\textsc{Jane Smith}\thanks{Corresponding author} \\[1ex] % Second author's name
%\normalsize University of Utah \\ % Second author's institution
%\normalsize \href{mailto:jane@smith.com}{jane@smith.com} % Second author's email address
}
\date{\today} % Leave empty to omit a date
\renewcommand{\maketitlehookd}{%
\begin{abstract}
\noindent While the Filecoin network has proven to successfully operate since 2020 and the launch of its mainnet,
the realization of its full potential is still in its early stages.
Among the leads for the near future, the configurability of its virtual machine for instance offers the promise of automated brokering systems.
This light paper tries to first show why the functioning of such a system cannot be considered as technologically nor politically neutral
and starts to lay down the first elements of the design of a coordinated and automated
brokerage layer on top of Filecoin.
After synthesizing the elements that can reasonably be fixed at the time of writing,
it provides a quick overview of the remaining limitations and open discussions before such a project can eventually be realized.
\end{abstract}
}

%----------------------------------------------------------------------------------------

\begin{document}

% Print the title
\maketitle

%----------------------------------------------------------------------------------------
%	ARTICLE CONTENTS
%----------------------------------------------------------------------------------------

\section{Introduction — Of the importance of digging the Web3 channel}

\lettrine[nindent=0em,lines=3]{S}{ince} the launch of Bitcoin in 2009, the types, promises, and uses of blockchains have diversified widely.
The emergence of the Ethereum blockchain and the Solidity smart contract language in 2015 for example undeniably marked a major milestone.

In parallel and for various motives such as the search for improved scalability,
different variants of these technologies have continually surfaced, making the deliberate choice
to cut back definitively on the openness and decentralization of the network, yet one of the elements of the so-called blockchain trilemma\footnote{The
blockchain trilemma refers to the practical problem met when looking for a balance between the three elements
expected to characterize a perfect blockchain: decentralization, security, and scalability.}.
Today this dichotomy is accentuated around the definitions and soundness of the notions of \emph{closed metaverse} and \emph{open metaverse}.
It is however undeniable that in the continuity of the development of the most open blockchain networks,
the potential of Web3 technologies is far from being fully explored.
Subjects of money, finance, or digital objects — with the advent non-fungible tokens (NFTs) — have probably already been scratched.
But if, as it is now often the case in the literature, the spectrum of their use is broadened to a much wider range of types of data,
by starting to consider NFTs as \emph{Web3 backpacks} that may be tied to any data stream,
then it becomes conceivable, and even necessary, to invest more seriously in the subjects of identity,
self-sovereign identity (SSI), and the storage of data associated with the experiences of these identities.

Alongside other protocols such as Arweave\footnote{\href{https://www.arweave.org/}{Arweave}
is a protocol focused on providing permanent storage.},
new networks that can be described as blockchain-enabled cooperative storage clouds have emerged,
most notably the \href{https://filecoin.io/}{Filecoin} blockchain, in conjunction with
\href{https://ipfs.io/}{the InterPlanetary File System (IPFS)}.
This undeniable technological breakthrough has now proven sound operation on its mainnet for more than a year and is storing
over 115 PiB of data\footnote{Numbers extracted by \href{https://file.app/}{file.app} analytics, on June 19th, 2022.}.
While the current version of the protocol constitutes in itself a major achievement and as attention could hazardily start to focus on relatively siloed uses,
exploiting above all a competitive advantage on the competitive pricing of large datasets archiving,
it seems at least equally important to continue to develop the potential of this network in more purely Web3 oriented uses.
In that sense, focusing on coordination functionalities in the network or interactions with other Web3 protocols represents both a challenge and an opportunity.
While since the end of 2021, the development of the new \href{https://fvm.filecoin.io/}{Filecoin Virtual Machine (FVM)} is pushing positively in this direction,
the present Morphée project aims at digging more precisely into the coordination topic and at crystallizing compelling ideas around one dedicated project.

%------------------------------------------------

\section{The hows and whys of Morphée}

\subsection{Initial rationale for a brokerage layer}

The Filecoin blockchain is a great tool to instill trust, without a third party,
between a user willing to store data on the IPFS network and a provider capable of fulfilling that request.
The blockchain allows a storage provider to prove that it initially received the data to be stored and to then continuously demonstrate,
throughout the duration of the contract, that it maintains the storage of this data at the expected location.
Otherwise, the machine would inevitably inflict an economic penalty.
This is essentially the real technological breakthrough that Filecoin has already achieved.

At present, therefore, buyers and providers maintain a large part of their exchanges off-chain.
The encounter of the storage buyer and the provider, the majority of the discussions before signing any deal, are indeed done off-chain.
However, the more the quality of service of the suppliers will be attested directly on-chain,
and the more in particular the reputation systems of the providers will be decentralized,
the easier it will become to increase the level of responsibility and intelligence available directly on the blockchain.
We could indeed relieve the customer of the burden of choosing their supplier manually,
and let them dictate this choice according to reputation metrics,
dictating the storage parameters with an adequate language and \emph{profiles} rather than directly specifying the precise list of their providers.
This should realize another step towards a truly open place of storage supply across the planet (or planets).

In short, this gain in expressiveness should initially serve clients to dictate their storage requirements
and let the trustless machine, over the course of time, choose the supplier(s) best able to fulfill them, rather than having to make their choices themselves.

\subsection{Primitives of the brokerage layer}

Formalizing this system requires some new lexical items.
We will say that providers describe and propose \emph{racks} while clients handle \emph{buckets}.
A bucket is characterized by its content, a rack by its capacity, and both by their configuration.

A configuration, real for a rack and expected for a bucket, is described according to a set of shared primitives
calibrated on the following characteristics:
\begin{itemize}
  \item geographic location
  \item latency and throughput performance, tied to the ability or will to store unsealed copies
  \item availability
  \item durability
  \item greenness, i.e. the ability to store data with a low carbon footprint
  \item providing an HTTP gateway
  \item pricing conditions
\end{itemize}

While these characteristics are fixed on the provider side, they pave the way on the storage buyer's for more
programmability of their storage configurations, with higher-level storage \emph{profiles}.
This gain in expressiveness should translate into plasticity of storage arrangements,
as for instance the set of providers meeting a customer's particular expectations may change over time in a way
ultimately transparent to the customer.
The notion of storage \emph{profiles} also makes it possible to consider certain new use cases
— such as setting geolocalized performance requests that would result in requirements for both performance
and the geographic distribution of replicas.

\subsection{Embracing the non-neutreality of the reconciliation algorithm}

Matching buyers' expectations with providers' supply over time is the logical responsibility of the reconciliation algorithm.
By introducing this component, the second main rationale behind Morphée is touched upon: the motives of this algorithm are not neutral.
Should it aim at minimizing customers' expenses?
Or on the contrary to maximize the global profits of suppliers?
Should it favor suppliers using decarbonized energy and if so to what extent?
Should it even economically favor the storage of certain data sets over others?

These are just the premises of real questions that need to be asked
and which answers could potentially apply to the entire network.
Answers that could be found, calculated by an evolutionary but unique network-wide reconciliation algorithm.

Morphée therefore proposes to embrace the necessary non-neutrality of such an algorithm,
which on the one hand makes possible a gain in expressiveness and intelligence useful to individuals' use of Filecoin,
and on the other hand allows to guide at the global scale the objectives of development and use of storage resources.
Therefore, Morphée chooses, from the outset, to make this algorithm configurable and to entrust its governance to the most legitimate community for this purpose.
This is the objective of Morphée, by providing the protocol, the network and the tools that unlock collaboration in order to jointly achieve both individual and collective objectives.



While this section has laid the groundwork for the design of Morphée,
the remainder of this document provides an attempt to describe the first technical elements that may now be outlined.

%------------------------------------------------

\section{Early technical design considerations}

\subsection{Optimizations based on the content of buckets}

In the chosen design, the very content of the buckets can be taken into consideration during the execution of the reconciliation algorithm.
This makes possible, for example, to globally favor the storage of certain datasets deemed to have higher priority,
or to factor out the storage costs of datasets or portions of datasets targeted by several clients at the same time.
While standards such as Content Identifiers (CIDs) and Interplanetary Linked Data (IPLD) should increase the potential impact on this part of the protocol,
the implementation details are yet to settle concerning in particular the availability of this metadata,
i.e. the availability of the information allowing to efficiently identify common parts in the content of different buckets.

Nevertheless, this problem can be considered as logically totally distinct from other optimization problems.
It can be considered as a pre-processing step to transform an initial set of buckets into a new set, optimized with respect to their content.


\subsection{Setting the reconciliation problem}

The issue of reconciling buyers' expectations and providers' capabilities is written as a combinatorial problem.
More precisely, it is written as an assignment problem which tasks and agents are parameterized from different sources:
\begin{itemize}
  \item the internal state of the machine, made of solutions of past iterations of the problem
  \item contextual information coming from the Filecoin mainnet, e.g. the status of tracked storage deals
  \item bucket profiles
  \item and rack profiles
\end{itemize}

\subsection{Around a solution to the reconciliation problem}

The reconciliation algorithm is in charge of solving a maximum generalized assignment problem.

It is known that a good approximation of the solution can be found in polynomial time. \cite{Fleischer:2006}

Initially, this algorithm should be able to rationally run on one (or more) sidechain(s) connected to the Filecoin mainnet via optimistic rollups.

%------------------------------------------------

\section{Conclusion}

The Morphée light paper lays the groundwork for what a coordinated and automated
brokerage layer could look like.
While the Filecoin blockchain has proven to work in production for more than a year,
it would now seem essential to look beyond its often siloed applications and to work on more collaborative uses of the network.

At the same time, many complementary projects are advancing and enriching an ecosystem that is increasingly capable
of realizing this complete vision, with inter-chain communication, layer-2 solutions, decentralized reputation
integrating first and foremost energy and ecological data, etc.

The mission of this light paper and the team originally behind it would be realized if it could help to structure
the debates that would lead to the emergence of a real coordinated and automated brokerage layer.
The mission would be fulfilled if one day such a layer arises that continuously and simultaneously
meets the objectives of each individual and those shared by the global network.


%----------------------------------------------------------------------------------------
%	REFERENCE LIST
%----------------------------------------------------------------------------------------

\begin{thebibliography}{99} % Bibliography - this is intentionally simple in this template

\bibitem[Fleischer, Lisa, et al., 2006]{Fleischer:2006}
Fleischer, Lisa, et al. (2006).
\newblock Tight approximation algorithms for maximum general assignment problems.
\newblock {\em Proceedings of the seventeenth annual ACM-SIAM symposium on Discrete algorithm.}

\end{thebibliography}

%----------------------------------------------------------------------------------------

\end{document}
